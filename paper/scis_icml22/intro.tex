\section{Introduction}

In high-dimensional settings, the information captured by the relatively few
labeled training samples is not sufficient for training estimators with good
predictive performance. Therefore, to select among the set of feasible
solutions, one needs to assume a certain structural bias for the estimator
% However, the learning problem becomes tractable if the ground truth has
% certain structural properties that are known a priori
(e.g.\ sparsity, invariance to rotations, translations etc).
For deep neural networks (DNNs), this inductive bias is implicit and recent work [TODO
cite chizat] has revealed that it favors ``sparse'' solutions, similar to an
$\ell_1$ penalty. Moreover, work by [TODO cite telgarsky etc] has shown that
minimizing an exponential loss (e.g.\ logistic, softmax) on separable data leads
to the solution that maximizes the minimum margin on the training set (we refer
to this as the \emph{min-margin solution}).

In order to better understand the behavior of overparameterized neural networks,
in this paper we study simple linear predictors that mirror the structure that
has been observed for DNNs. In particular, we focus on binary classification
with a linear and sparse ground truth. We study the interpolator that maximizes
the min-$\ell_1$-margin (i.e.\ basis pursuit), since a small $\ell_1$ norm is
known to induce sparsity [TODO cite tibshirani etc].

Despite a large trove of positive results for maximum min-margin interpolation
in high dimensions [TODO cite hastie, guillaume, etc], a recent work [TODO cite
konstantin] finds that predictors obtained with basis pursuit tend to exhibit
larger variance in high dimensions. Despite using the right structural bias,
maximizing the min-$\ell_1$-margin can even lead to larger error than its
$\ell_2$ counterpart solely due to this increase in variance. This observation
raises the following natural question:

\begin{center}

\emph{Can we achieve lower variance while still exploiting the sparsity-inducing
bias of $\ell_1$ interpolation?}

\end{center}

In this paper, we propose to reduce the variance by minimizing the
\emph{average}-$\ell_1$-margin instead of the \emph{min}-$\ell_1$-margin.
Intuitively, in high-dimensions, the average-margin solution is determined by
all training points, unlike the min-margin interpolator. This intuition is
depicted in the sketch in Figure \at{TODO}. Our
results indicate that, in high dimensions, average-margin interpolation does
indeed lead to lower variance, and hence, lower estimation error. Surprisingly,
we our experiments reveal that the average-margin solution also has lower bias,
compared to the min-margin interpolator.  Finally, in Section~\at{TODO} we argue
that despite its benefits in high-dimensions, the average-$\ell_1$-margin
estimator cannot readily replace min-margin interpolation in all settings. In
particular, we show that the average-margin solution is less robust to outliers
in low dimensions.





