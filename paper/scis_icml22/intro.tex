\section{Introduction}

In high-dimensional settings, the information captured by the relatively few
labeled training samples is not sufficient for training estimators with good
predictive performance. Therefore, to select among the set of feasible
solutions, one needs to assume a certain structural bias for the estimator
% However, the learning problem becomes tractable if the ground truth has
% certain structural properties that are known a priori
(e.g.\ sparsity, invariance to rotations, translations etc).
For deep neural networks (DNNs), this inductive bias is implicit and recent work [TODO
cite chizat] has revealed that it favors ``sparse'' solutions, similar to an
$\ell_1$ penalty. Moreover, work by [TODO cite telgarsky etc] has shown that
minimizing an exponential loss (e.g.\ logistic, softmax) on separable data leads
to the solution that maximizes the minimum margin on the training set (we refer
to this as the \emph{min-margin solution}).

In order to better understand the behavior of overparameterized neural networks,
in this paper we study simple linear predictors that mirror the structure that
has been observed for DNNs. In particular, we focus on binary classification
with a linear and sparse ground truth. We study the interpolator that maximizes
the min-$\ell_1$-margin (i.e.\ basis pursuit), since a small $\ell_1$ norm is
known to induce sparsity [TODO cite tibshirani etc].

Despite a large trove of positive results for maximum min-margin interpolation
in high dimensions [TODO cite hastie, guillaume, etc], we find that predictors
obtained with basis pursuit 

despite the many positive results for min-l2 margin interpolation, we find that for l1 the error of the estimator deteriorates as d/n increases (in line with konstantin’s paper)

we find that this increase in error for min l1-margin is due to higher variance (since the bias is well aligned with the structure of the GT)

we propose to reduce the variance with avg l1-margin interpolation and find that
indeed it helps reduce variance in high dimensions; moreover, surprisingly, we
see that it also reduces the bias

finally, we discuss limitations of the avg l1-margin solution in low dimensions



